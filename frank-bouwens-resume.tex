% LaTeX source of my resume
% =========================

% Commented for easy reuse... ;)

% See the `README.md` file for more info.

% This file is licensed under the CC-NC-ND Creative Commons license.


% start a document with the here given default font size and paper size
\documentclass[10pt,a4paper]{article}

% include the `tex` instructions that takes care of loading packages and defining commands
\include{resume-commands}



\begin{document}  % begin the content of the document
\sloppy  % this to relax whitespacing in favour of straight margins

\maintitle{Frank Bouwens}{May 30, 1991}  % title on top of the document

\nobreakvspace{0.3em}  % add some page break averse vertical spacing

% \noindent prevents paragraph's first lines from indenting
% \mbox is used to obfuscate the email address
% \sbull is a spaced bullet
% \href well..
% \\ breaks the line into a new paragraph
\noindent\href{mailto:frankkie12345@gmail.com}{frankkie12345\mbox{}@\mbox{}gmail.com}\sbull
\textsmaller{+}31.641594567\sbull
frankkie12345 \emph{(Skype)}\sbull
\href{http://www.linkedin.com/in/frankkie12345}{www.linkedin.com/in/frankkie12345}
\\
Harp 29\sbull
3068\thinspace {\sc HM}\sbull
Rotterdam\sbull
The Netherlands

\spacedhrule{0.9em}{-0.4em}  % a horizontal line with some vertical spacing before and after

\roottitle{Summary}  % a root section title

\vspace{-1.3em}  % some vertical spacing
\begin{multicols}{2}  % open a multicolumn environment
\noindent \emph{Leergierige developer met passie voor mobiele applicaties en open source software.}
\\
\\
Op de basisschool begon ik al met programmeren, namelijk met het programma Game Maker van \href{http://www.yoyogames.com}{yoyogames.com} (toen nog gamemaker.nl). De programmeertaal die ik daarin gebruikte, GML (Game Maker Language) was makkelijk om te leren. Het is namelijk een object-georienteerde taal met veel mogelijkheden die andere populaire programmeertalen ook ondersteunen. Een goede taal om mee te leren en te beginen met programmeren.
\\
\\
Op de middelbare school kwam ik in aanraking met PHP. Dat was heel anders dan GML. Maar wel anders op een goede manier. Ik heb vele uren met PHP en MySQL kunnen experimenteren. Het heeft niet veel echte websites opgeleverd, maar wel talloze proof-of-concept pagina's. Ik gebruik PHP regelmatig om websites dynamisch te maken en om REST-functies mee aan te sturen.
\\
\\
Op de hogeschool werd mij geleerd in Java te programmeren. Dit paste bij mij als een handschoen. Java werkt object georienteerd net als GML (en voor een deel ook PHP). Ik kon meteen aan de slag. Op dit moment kwam ik voor het eerst echt in aanraking met Threads. Iets waar ik nu dagelijks gebruik van maak.
\\
\\
Later op de hogeschool kwam er een project voorbij waarbij een Android applicatie moest worden ontwikkeld. Ik was direct verkocht.
Sindsdien werk ik veel met Android, zowel als gebruiker als ontwikkelaar.
Ik heb inmiddels talloze applicaties ontwikkeld, waarvan enkele te vinden zijn in Google Play (de voormalige Android Market).
\\
\\
Bij bedrijven waar ik heb gewerkt heb ik ervaring opgedaan met onder andere Laravel (PHP framework) en Swift voor het ontwikkelen van iOS applicaties. Verder heb ik recentelijk kennis gemaakt met ReactJS. 
\end{multicols}

\spacedhrule{0em}{-0.4em}

\roottitle{Ervaring}

\headedsection
  {\href{https://www.bitfactory.nl}{Bitfactory}}
  {\textsc{Rotterdam, Nederland}} {
  
    \headedsubsection
      {\acr{PHP, Android en iOS Developer}}
      {Mrt 2017-Mrt 2019}
      {\bodytext{Bitfactory is een digital agency die voornamelijk werkt aan maatwerk websites. Meerdere van deze websites (o.a. Dierenbescherming en Tony's Chocolonely) hebben de 'Website van het jaar'-award ontvangen. De websites bij Bitfactory worden allemaal gemaakt in PHP, maar er wordt gewerkt met diverse frameworks. 
      \\
      Zo heb ik gewerkt aan systeem (BijZijn) om aanwezigheid van studenten bij cursussen. Dit bestaat uit een Web-portaal en een API, ontwikkeld met respectievelijk Laravel en Lumen; En mobiele apps voor Android (Java) en iOS (Swift). BijZijn is gekoppeld aan CRM-systemen die gebouwd zijn CakePHP, ook ontwikkeld bij Bitfactory. 
      \\
      In opdracht van de Gemeente Rotterdam heeft Bitfactory een systeem ontwikkeld voor buitengewoon opsporingsambtenaren (BOA's, stadswachten), bestaande uit een web-portaal, API en apps voor Android en iOS. De BOA's gebruiken deze apps om onderling te communiceren en rapporteren.
      \\
      Verder heb ik een iOS applicatie ontwikkeld voor InterHear. Dit is een applicatie die kan dienen als gehoorapparaat voor die mensen die nog niet slecht genoeg horen voor een duur gehoorapparaat maar wel ondersteuning kunnen gebruiken. Geluid wordt door de iPhone of iPad versterkt en spraak wordt verstaanbaarder gemaakt. Ideaal voor tijdens vergaderingen in rumoerige plaatsen.
      }
      }
  }

\headedsection
  {\href{https://www.innovattic.nl}{Innovattic}}
  {\textsc{Delft, Nederland}} {
  
    \headedsubsection
      {\acr{Android Developer}}
      {Okt 2016-Mrt 2017}
      {\bodytext{Innovattic werkt aan diverse projecten voor mobiele applicaties (voor o.a. healthcare) en serious games. Bij Innovattic heb ik gewerkt aan een Android applicatie met behulp van ResearchStack (een port van Apple's ResearchKit) die bedoeld is om bij patienten dagelijks de gezondheid te peilen. Als er verontrustende waardes worden ingevuld kan direct de dokter worden gewaarschuwd. Door de historische waarden in te zien kan een dokter zien of patienten goed genezen zonder vaak langs te moeten gaan.
      \\
       Voor Kids Worldwide Factory heeft Innovattic een tablet applicatie (Android en iOS) ontwikkeld voor kleine kinderen. De app Kidiyo, bevat leerzame spelletjes en afleveringen van de kinder-series 'Chloe's Closet' en 'Princess Arabelle'. 
      }
      }
  }

\headedsection
  {\href{https://www.auxilium.nl}{Auxilium}}
  {\textsc{Delft, Nederland}} {
  
    \headedsubsection
      {\acr{Android en .NET Developer}}
      {Sep 2015-Okt 2016}
      {\bodytext{Auxilium neemt projecten aan van verschillende branches. Bij Auxilium heb ik o.a. gewerkt aan de Android applicatie van AutoReset. Deze app wordt gebruikt door chauffeurs die in dienst zijn bij AutoReset. Via de app kan de rittenplanning en registratie worden gedaan, verder kunnen ook eventuele onkosten (zoals tankbeurten en wassen) direct via de app worden gedeclareerd. 
      \\
      Verder heb ik gewerkt aan een applicatie voor patienten van ziekenhuizen. De patienten kunnen komende afspraken inzien en op basis hiervan aanvullende diensten aanschaffen. Zoals het inhuren van een chauffeur die de patient thuis ophaalt, mee gaat met bezoek aan de dokter en vervolgens de patient weer veilig thuis brengt. Daarnaast heb ik gewerkt aan een aantal zakelijke .NET webapplicaties. Zo heb ik gewerkt aan een planning systeem voor ouderenorganistaties. De stichtingen hebben busjes die door vrijwilligers worden gebruikt om ouderen van huis naar de supermarkt te brengen, maar ook naar diverse uitjes.}
      }
  }


\headedsection  % sets the header for the section and includes any subsections
  {\href{http://www.veliq.com}{VeliQ}}
  {\textsc{Barendrecht, Nederland}} {%

  \headedsubsection  % sets the header for a subsection and contains usually body text
    {\acr{Android Developer}}
    {Feb 2012-Aug 2015}
    {\bodytext{VeliQ maakt Mobile Device Management oplossingen voor de professionele markt. In MobiDM worden diverse andere system geintegreed om het systeem aantrekkelijk te maken voor grote bedrijven.
\\
Tijdens mij afstudeerstage bij VeliQ heb ik een API ontwikkeld waarmee andere bedrijven een mobiele beheerders portal kunnen maken. Deze was gebaseerd op een bestaande API binnen het pakket maar dan beter geschikt voor op mobiele apparaten. Zo is gebruik gemaakt van REST/JSON in plaat van SOAP. SOAP-calls zijn een stuk zwaarder, er zijn geen goede libraries voor en verbruiken meer data-verkeer. Ik heb proof-of-concept applicaties ontwikkeld om aan te tonen dat de API correct functioneerd. 
\\
Naast mijn afstudeerstage heb ik bij VeliQ ook gewerkt aan de Android client voor het systeem. Deze applicatie wordt gebruikt om de toestellen te beheren. De app dwingt op het toestel security-policies af. Zoals het verplicht stellen van een lockscreen-password. Ik heb onder andere gewerkt aan het inbouwen van een anti-virus module en een simpelere / gebruiksvriendelijkere  enrollment-procedure. }
   }
}

\headedsection
{\href{http://www.anewspring.nl}{aNewSpring}}
{\textsc{Rotterdam, Nederland}}
{
 \headedsubsection
 {\acr{Android Developer}}
 {Feb 2011}
 {\bodytext{aNewSpring maakt e-learning oplossingen voor opleiders. Het is een volledig systeem om cursussen te geven en studenten te begeleiden. Bij dit bedrijf heb ik een mobiele applicaties ontwikkeld die cursisten helpt met leren. De applicatie geeft de cursist iedere dag 10 vragen over de cursus. Vragen die incorrect zijn beantwoord komen de volgende dag terug. Net zolang to alle vragen correct zijn beantwoord. En zijn verschillende vraagtypes ondersteund, zoals multiple choice (\'{e}\'{e}n correct antwoord), multi choice (meerdere antwoorden), text (een woord is het antwoord) en hotspot (wijs plek aan op plaatje). De applicatie werkt samen met de backend-system van aNewSpring om dit mogelijk te maken. Tijdens het ontwikkelen van deze app heb ik samengewerkt met een andere Android programmeur en meerdere backend programmeurs.}}
}

\headedsection
{\href{http://www.themobilecompany.nl}{The Mobile Company}}
{\textsc{Amsterdam, Nederland}}
{
 \headedsubsection
 {\acr{Android Developer}}
 {Jan 2009-Jan 2011}
 {\bodytext{The Mobile Company ontwikkelt mobiele applicatie in opdracht van andere bedrijven. Ik heb meegeholpen bij Android projecten als: iTour360 en 9292ov Pro.
\\
iTour306 is een locatie gebaseerde tour-guide app. De app geeft namelijk een rondleiding door een stad, in de eerste versie was dat alleen Almere. De app vraagt de gebruiker om naar een bepaalde lokatie te lopen, om vervolgens over diverse bezienswaardigheden informatie voor te lezen. Toen deze app uitkwam waren smartphones nog niet zo heel gewoon. Niet iedereen had er een op zak. Om die reden werd app voorgeinstalleerd op HTC Wildfire toestellen. Om die toestellen vervolgens verhuren bij een VVV-kantoor. Om misbruik te voorkomen mocht de gebruiker niet buiten de applicatie komen. Dit is gedaan door een custom homescreen op de toestellen te zetten. Via dit homescreen kan alleen de iTour360 app weer worden geopend.
\\
Toen ik aan de 9292ov Pro app ging werken was deze net uitgebracht. De ontwikkelaar die de app gemaakt had was toen net weg bij het bedrijf, dit was namelijk een afstudeerder. Ik kreeg alleen de broncode. Er zaten veel bugs in het programma en die moesten eruit. Verder moesten er nieuwe functies worden ingebouwd. Het heeft een tijdje geduurd om code te begrijpen. Dit was omdat de ontwikkelaar geen Android app had ontwikkeld, maar een iOS app. En die broncode was door een converter gehaald om er een Android app van te maken. Hierdoor ontstonden onder andere methodes van honderden regels lang, waarbij de code na 16 indents pas begon. Na enkele maanden waren de belangrijkste bugs eruit, was een vertrektijden scherm ingebouwd en was er een widget toegevoegd.
}}

}

\begin{center}
  \emph{Zie ook \href{http://www.linkedin.com/in/frankkie12345}{mijn profiel op Linkedin} voor een complete lijst van mijn werkervaring en aanbevelingen.}
\end{center}


\spacedhrule{-0.2em}{-0.4em}

\roottitle{Scholing}

\headedsection
  {Hogeschool Rotterdam}
  {\textsc{Rotterdam, Nederland}} {%
  \headedsubsection
    {Behaald Bachelor informatica}
    {2008 -- 2014}
    {\bodytext{Programmeren in Java, Android applicaties, etc.}}
}

\headedsection
  {Comenius College}
  {\textsc{Capelle aan den IJssel, Nederland}} {%
  \headedsubsection
    {\acr{HAVO} \textnormal{(Middelbare school, profiel: Natuur en Techniek)}}
    {2003 -- 2008} {}
}

\headedsection
  {Udacity}
  {\textsc{Udacity.com}} {
  \headedsection
  {\acr{Android} \textnormal{Online cursus Android apps ontwikkelen}}
  {2014 -- 2015} 
  {\bodytext{Door de Dutch Android User Group (GDG) uitgekozen om als een van de eerste een door Google ontwikkelde Android cursus te volgen. Daarna deze cursus, als leraar, weer gegeven aan anderen.}}
  }

\spacedhrule{0.5em}{-0.4em}

\roottitle{Vaardigheden}

\inlineheadsection  % special section that has an inline header with a 'hanging' paragraph
  {Technische specialisaties:}
  {Software design en implementatie (in een team). \\
  Ik heb veel ervaring met de programmeertalen: \acr{Java} en \acr{PHP}. \\
  Daarnaast heb ik in mindere mate ervaring met: \acr{Kotlin} (Android), \acr{Swift} (iOS), \acr{TypeScript} en \acr{JavaScript}. \\
  Verder heb ik gewerkt met technologie\"en als:\ \acr{HTML+CSS}, \acr{XML}, \acr{REST}, \acr{JSON}, \acr{SOAP}, \acr{Vagrant}, en \acr{MySQL}.}

\inlineheadsection
  {Talen:}
  {Nederlands \emph{(Moedertaal)}, Engels \emph{(Vloeiend)}, Duits \emph{(Beginner)}.}


\spacedhrule{1.6em}{-0.4em}

\roottitle{Intresses}

\inlineheadsection
  {In alfabetische volgorde:}
  {Android, bronies, cryptografie, Google, muziek, open source, politiek, ruimtevaart, software engineering, typograpfie (\LaTeX).}

\spacedhrule{1.6em}{-0.4em}

\roottitle{Projecten}

\inlineheadsection{Udacity Android cursus}{
    Google heeft samen met Udacity een cursus ontwikkeld, waarbij de studenten leren om een Android app te ontwikkelen. Deze cursus werd via de Dutch Android User Group (lokale Google Developer Group) aan mij aangeboden. Nadat ik deze cursus met goed gevolg heb voltooid, heb ik als docent deze cursus geven aan een klas van andere studenten. Bij het uitvoeren van deze cursus hoort ook het ontwikkelen van een app, waarin de diverse technieken die in de cursus zaten worden gebruikt in de praktijk. Mijn \emph{final project app} was de \emph{Convention app}. Een programmaboekje om te gebruiken tijdens een conventie.
}

\inlineheadsection{Convention, programmaboekje voor HWcon}{
Als onderdeel van de Udacity Android cursus moest ik een App ontwikkelen. Ik heb ervoor gekozen om een App te ontwikkelen waarin het programmaboekje van een conventie kan worden bekeken. De app is opgezet dat het vrij makkelijk kan worden aangepast om bij een andere conventie te worden gebruikt. De applicatie is ontwikkeld met \href{http://www.hwcon.nl/}{HWcon} in het achterhoofd. Deze conventie werd gehouden op 20 en 21 feburari 2015.
\\
Bij deze applicatie heb ik onder andere gebruik gemaakt van Java, XML en SQLite voor de clientkant en PHP, MySQL voor de serverkant en REST/JSON voor de communicatie daartussen.
De applicatie maakt gebruik van Material Design, zoals gebruikelijk bij Android vanaf versie 5.0. Ook is er gebruik gemaakt van een Content-Provider, dit is een design-pattern die in de cursus werd aanbevolen. De \href{https://github.com/frankkienl/Convention}{broncode} is te vinden op Github.
}

\inlineheadsection
{The Next Question (Woordspelletje voor met vrienden)}
{Hobby project, Firebase en ReactJS. Op diverse feestjes heb ik met vrienden woordspellen gespeeld zoals Cards Against Humanity, Quiplash en WordsGame. Deze spellen zijn heel leuk en gezellig maar hebben allemaal als nadeel dat je hiervoor een flinke bak met kaartjes of een laptop moet meenemen. Met 'The Next Question', kan met smartphones worden gespeeld. Spelers krijgen vragen, later kunnen andere spelers stemmen wie het beste (of grappigste) antwoord heeft gegeven. Bij dit project stap ik behoorlijk buiten mijn comfort-zone door gebruik te maken van web-technieken in plaats van het maken van een app. De backend is geschreven in TypeScript, (JavaScript) en draait op Firebase Functions. De frontend is geschreven in ReactJS. Verder wordt gebruik gemaakt van Firebase Hosting en de Firestore als database. Het spel is nog niet voltooid, maar het is al wel speelbaar. \href{https://frankkienl-tnq.firebaseapp.com/}{TNQ}
}

\inlineheadsection
{Tux Words (Lingo):}
{Hobby project. De app laat de gebruiker \emph{Lingo} spelen, zoals op TV. Deze applicatie is ontwikkeld om dat een vergelijkbare app niet op alle Android apparaten werkt. De concurrerende app werkt namelijk alleen op toestellen met een mdpi of hoger scherm. Kleine (vaak goedkopere) toestellen met een ldpi scherm zijn daarmee uitgesloten. Tux Words is gemaakt om op zoveel mogelijk apparaten te kunnen werken. Om die reden werkt het op alle geteste smartphones en tablets, maar ook op Google TV. Tux Words is mijn meest populaire app, met meer dan 20.000 downloads op de Google Play Store. Verder is deze app ook te vinden in de \href{http://apps.samsung.com/venus/topApps/topAppsDetail.as?productId=000000498917}{Samsung Apps Store} en de \href{http://www.amazon.com/FrankkieNL-Tux-Words-Lingo/dp/B0096M4AYU/}{Amazon Appstore}}

\inlineheadsection
{BAXY Launcher}
{Hobby project, custom homescreen voor OUYA game console. De \emph{OUYA} laat de gebruiker Android games spelen op een TV met behulp van een controller. Standaard toont OUYA de lijst van spellen in een niet te voorspellen volgorde. De volgorde wordt namelijk gebaseerd op volgorde van installatie, laatste update, laatst gebruikt en meest gebruikt. Het komt erop neer dat elke keer als een game wordt gestart, de volgorde van de lijst veranderd. Dit vinden de meeste gebruikers vervelend. Om die reden heb ik BAXY Launcher ontwikkeld. Deze zet de lijst van spellen namelijk in alfabetische volgorde. De launcher is verder uitgebreid met mogelijkheiden om te filteren. In tegenstelling tot de standaard launcher kan BAXY ook widgets weergeven en gebruikmaken van een livewallpaper. Zie \href{http://ouyaforum.com/showthread.php?4436-BAXY-Custom-Launcher}{ouyaforum.com/showthread.php?4436} voor meer informatie en screenshots.}


\inlineheadsection
  {ParcDroid:}
  {Schoolproject, mijn eerste Android applicatie. ParcDroid is een navigatie applicatie voor kinderen in \emph{De Efteling}. Er kan van tevoren een verzamelplaats worden gekozen. Als het kind de weg kwijt is geraakt helpt deze app om de weg naar deze lokatie te vinden. De app gebruikt daarvoor onder andere GPS en het kompas. Dit project heeft mij niet alleen meer over Android geleerd, maar ook over wiskunde en het Dijkstra-algoritme. (Het algoritme waarmee een route kan worden bepaald, wordt ook gebruikt door onder andere TomTom navigatiesystemen) ParcDroid heeft de eerste prijs in de wacht gesleept en was daarmee de beste applicatie van dat schooljaar.}
  

\inlineheadsection{Google Glass}{
Via een collega heb ik een week lang met \emph{Google Glass} mogen experimenteren. In deze week heb ik een applicatie ontwikkeld die gebruik maakt van de camera en gezichten detecteerd. De applicatie herkent niet \emph{wie} hij ziet, maar alleen dat er een gezicht is.
}

\inlineheadsection{Apps voor Omate Truesmart Smartwatch}{
Via een \href{https://www.kickstarter.com/projects/omate/omate-truesmart-water-resistant-standalone-smartwa}{Kickstarter} project ben ik aan een \href{http://www.omate.com/}{Omate} Smartwatch gekomen. Ik kwam er al snel achter dat het horloge wat gebreken had. Met behulp van apps heb ik de voor mij belangrijkste problemen opgelost. Het horloge telt niet door als hij uit staat. Iets wat redelijk vaak gebeurt omdat de accu vrij snel leeg gaat. Gaat het horloge om 12:00 uit, en wordt hij om 13:00 weer aangezet, dan meld het horloge dat het 12:00 is. Het horloge heeft wel ingebouwde mogelijkheden om de tijd te synchroniseren, maar deze vereisen een simkaart of een sterk gps signaal. Om die reden heb ik een applicatie ontwikkeld die de tijd synchroniseerd via een NTP-server op het internet. Een ander probleem van het horloge is het ontbreken van volume-knoppen. Het horloge bevat wel een mp3-speler applicatie, maar die staat altijd even hard. Een simpele applicatie met volume-knoppen lost dit probleem op.
}

\begin{center}
  \emph{Zie ook \href{http://frankkienl.github.com/}{Github} en \href{https://play.google.com/store/apps/developer?id=FrankkieNL}{de Google Play Store} voor een uitgebreidere lijst van mijn projecten.}
\end{center}

\end{document}
